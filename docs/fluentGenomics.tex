% Options for packages loaded elsewhere
\PassOptionsToPackage{unicode=true}{hyperref}
\PassOptionsToPackage{hyphens}{url}
%
\documentclass[
  9pt,
  a4paper,
]{extarticle}
\usepackage{lmodern}
\usepackage{amssymb,amsmath}
\usepackage{ifxetex,ifluatex}
\ifnum 0\ifxetex 1\fi\ifluatex 1\fi=0 % if pdftex
  \usepackage[T1]{fontenc}
  \usepackage[utf8]{inputenc}
  \usepackage{textcomp} % provides euro and other symbols
\else % if luatex or xelatex
  \usepackage{unicode-math}
  \defaultfontfeatures{Scale=MatchLowercase}
  \defaultfontfeatures[\rmfamily]{Ligatures=TeX,Scale=1}
\fi
% Use upquote if available, for straight quotes in verbatim environments
\IfFileExists{upquote.sty}{\usepackage{upquote}}{}
\IfFileExists{microtype.sty}{% use microtype if available
  \usepackage[]{microtype}
  \UseMicrotypeSet[protrusion]{basicmath} % disable protrusion for tt fonts
}{}
\makeatletter
\@ifundefined{KOMAClassName}{% if non-KOMA class
  \IfFileExists{parskip.sty}{%
    \usepackage{parskip}
  }{% else
    \setlength{\parindent}{0pt}
    \setlength{\parskip}{6pt plus 2pt minus 1pt}}
}{% if KOMA class
  \KOMAoptions{parskip=half}}
\makeatother
\usepackage{xcolor}
\IfFileExists{xurl.sty}{\usepackage{xurl}}{} % add URL line breaks if available
\IfFileExists{bookmark.sty}{\usepackage{bookmark}}{\usepackage{hyperref}}
\hypersetup{
  pdftitle={Fluent genomics with plyranges and tximeta},
  pdfauthor={Stuart Lee, Di Cook, Michael Lawrence, Michael Love},
  hidelinks,
}
\urlstyle{same} % disable monospaced font for URLs
\usepackage[margin=1in]{geometry}
\usepackage{color}
\usepackage{fancyvrb}
\newcommand{\VerbBar}{|}
\newcommand{\VERB}{\Verb[commandchars=\\\{\}]}
\DefineVerbatimEnvironment{Highlighting}{Verbatim}{commandchars=\\\{\}}
% Add ',fontsize=\small' for more characters per line
\usepackage{framed}
\definecolor{shadecolor}{RGB}{248,248,248}
\newenvironment{Shaded}{\begin{snugshade}}{\end{snugshade}}
\newcommand{\AlertTok}[1]{\textcolor[rgb]{0.94,0.16,0.16}{#1}}
\newcommand{\AnnotationTok}[1]{\textcolor[rgb]{0.56,0.35,0.01}{\textbf{\textit{#1}}}}
\newcommand{\AttributeTok}[1]{\textcolor[rgb]{0.77,0.63,0.00}{#1}}
\newcommand{\BaseNTok}[1]{\textcolor[rgb]{0.00,0.00,0.81}{#1}}
\newcommand{\BuiltInTok}[1]{#1}
\newcommand{\CharTok}[1]{\textcolor[rgb]{0.31,0.60,0.02}{#1}}
\newcommand{\CommentTok}[1]{\textcolor[rgb]{0.56,0.35,0.01}{\textit{#1}}}
\newcommand{\CommentVarTok}[1]{\textcolor[rgb]{0.56,0.35,0.01}{\textbf{\textit{#1}}}}
\newcommand{\ConstantTok}[1]{\textcolor[rgb]{0.00,0.00,0.00}{#1}}
\newcommand{\ControlFlowTok}[1]{\textcolor[rgb]{0.13,0.29,0.53}{\textbf{#1}}}
\newcommand{\DataTypeTok}[1]{\textcolor[rgb]{0.13,0.29,0.53}{#1}}
\newcommand{\DecValTok}[1]{\textcolor[rgb]{0.00,0.00,0.81}{#1}}
\newcommand{\DocumentationTok}[1]{\textcolor[rgb]{0.56,0.35,0.01}{\textbf{\textit{#1}}}}
\newcommand{\ErrorTok}[1]{\textcolor[rgb]{0.64,0.00,0.00}{\textbf{#1}}}
\newcommand{\ExtensionTok}[1]{#1}
\newcommand{\FloatTok}[1]{\textcolor[rgb]{0.00,0.00,0.81}{#1}}
\newcommand{\FunctionTok}[1]{\textcolor[rgb]{0.00,0.00,0.00}{#1}}
\newcommand{\ImportTok}[1]{#1}
\newcommand{\InformationTok}[1]{\textcolor[rgb]{0.56,0.35,0.01}{\textbf{\textit{#1}}}}
\newcommand{\KeywordTok}[1]{\textcolor[rgb]{0.13,0.29,0.53}{\textbf{#1}}}
\newcommand{\NormalTok}[1]{#1}
\newcommand{\OperatorTok}[1]{\textcolor[rgb]{0.81,0.36,0.00}{\textbf{#1}}}
\newcommand{\OtherTok}[1]{\textcolor[rgb]{0.56,0.35,0.01}{#1}}
\newcommand{\PreprocessorTok}[1]{\textcolor[rgb]{0.56,0.35,0.01}{\textit{#1}}}
\newcommand{\RegionMarkerTok}[1]{#1}
\newcommand{\SpecialCharTok}[1]{\textcolor[rgb]{0.00,0.00,0.00}{#1}}
\newcommand{\SpecialStringTok}[1]{\textcolor[rgb]{0.31,0.60,0.02}{#1}}
\newcommand{\StringTok}[1]{\textcolor[rgb]{0.31,0.60,0.02}{#1}}
\newcommand{\VariableTok}[1]{\textcolor[rgb]{0.00,0.00,0.00}{#1}}
\newcommand{\VerbatimStringTok}[1]{\textcolor[rgb]{0.31,0.60,0.02}{#1}}
\newcommand{\WarningTok}[1]{\textcolor[rgb]{0.56,0.35,0.01}{\textbf{\textit{#1}}}}
\usepackage{longtable,booktabs}
% Allow footnotes in longtable head/foot
\IfFileExists{footnotehyper.sty}{\usepackage{footnotehyper}}{\usepackage{footnote}}
\makesavenoteenv{longtable}
\usepackage{graphicx,grffile}
\makeatletter
\def\maxwidth{\ifdim\Gin@nat@width>\linewidth\linewidth\else\Gin@nat@width\fi}
\def\maxheight{\ifdim\Gin@nat@height>\textheight\textheight\else\Gin@nat@height\fi}
\makeatother
% Scale images if necessary, so that they will not overflow the page
% margins by default, and it is still possible to overwrite the defaults
% using explicit options in \includegraphics[width, height, ...]{}
\setkeys{Gin}{width=\maxwidth,height=\maxheight,keepaspectratio}
\setlength{\emergencystretch}{3em} % prevent overfull lines
\providecommand{\tightlist}{%
  \setlength{\itemsep}{0pt}\setlength{\parskip}{0pt}}
\setcounter{secnumdepth}{-\maxdimen} % remove section numbering
% Redefines (sub)paragraphs to behave more like sections
\ifx\paragraph\undefined\else
  \let\oldparagraph\paragraph
  \renewcommand{\paragraph}[1]{\oldparagraph{#1}\mbox{}}
\fi
\ifx\subparagraph\undefined\else
  \let\oldsubparagraph\subparagraph
  \renewcommand{\subparagraph}[1]{\oldsubparagraph{#1}\mbox{}}
\fi

% Set default figure placement to htbp
\makeatletter
\def\fps@figure{htbp}
\makeatother

% https://github.com/rstudio/rmarkdown/issues/337
\let\rmarkdownfootnote\footnote%
\def\footnote{\protect\rmarkdownfootnote}

% https://github.com/rstudio/rmarkdown/pull/252
\usepackage{titling}
\setlength{\droptitle}{-2em}

\pretitle{\vspace{\droptitle}\centering\huge}
\posttitle{\par}

\preauthor{\centering\large\emph}
\postauthor{\par}

\predate{\centering\large\emph}
\postdate{\par}
\usepackage[]{natbib}
\bibliographystyle{plainnat}

\title{Fluent genomics with plyranges and tximeta}
\author{Stuart Lee, Di Cook, Michael Lawrence, Michael Love}
\date{}

\begin{document}
\maketitle
\begin{abstract}
In this workflow, we use the R/Bioconductor packages \emph{tximeta} and \emph{plyranges} for integrating results from an experiment using RNA-seq and ATAC-seq data. The \emph{tximeta} package provides an automated way of importing RNA-seq transcript quantifications. It does this by creating a \emph{SummarizedExperiment} object that knows the correct reference transcriptome and the metadata associated with the quantification software. Since the quantifications are transformed into a \emph{SummarizedExperiment} object, they are ready for downstream analysis using many other Bioconductor packages. The \emph{plyranges} package provides a framework for expressing operations on genomic coordinates such as finding overlaps and performing aggregations and genomic arithmetic. Here it is used to perform sensitivity analysis on the results of a differential expression and differential accessibility analysis with respect to their genomic context.
\end{abstract}

\hypertarget{introduction}{%
\section{Introduction}\label{introduction}}

To begin we will examine a subset of the RNA-seq and ATAC-seq data from \citet{alasoo}
- \href{https://doi.org/10.1038/s41588-018-0046-7}{doi: 10.1038/s41588-018-0046-7}.
The experiment involved treatment of macrophage cell lines from a number of
human donors with interferon (IFN) gamma, \emph{Salmonella} infection, or both
treatments combined. In the original study, the authors examined gene
expression and chromatin accessibility in a subset of 86 successfully
differentiated induced pluripotent stem cells (iPSC) lines, and examined
baseline quantitative trait loci (QTL) and response QTL for both expression and
accessibility. In the original study, it was found that there were expression
QTLs that had an effect on chromatin accessibility in naive macrophage cells
upon IFNg stimulation and consequently the variants implicated likely result in
changes in binding of cell-specific transcription factors.

In this workflow, we will perform a much simpler analysis than the one found in
\citet{alasoo}, using their publicly available RNA-seq and ATAC-seq data (ignoring the
genotypes). We will examine the effect of IFNg stimulation on gene expression
and chromatin accessibility, and look to see if there is an enrichment of
differentially accessible (DA) ATAC-seq peaks in the vicinity of differentially
expressed (DE) genes. This is plausible, as the transcriptomic response to IFNg
stimulation may be mediated through binding of regulatory proteins to
accessible regions, and this binding may increase the accessibility of those
regions such that it can be detected by ATAC-seq.

Throughout the workflow, we will use existing Bioconductor infrastructure to
understand these datasets. In particular, we will emphasize the use of the
Bioconductor packages \emph{plyranges} and \emph{tximeta}. The first package is be used
to perform easily-readable transformations of data tied to genomic ranges, e.g.
shifts, windows, overlaps, etc. The \emph{plyranges} package is described by
\citet{Lee2019}, and leverage underlying range operations described by \citet{granges}. The
second package described by \citet{Love2019-tximeta} is used to read RNA-seq
quantification into R/Bioconductor, such that the genomic ranges are
automatically attached to the quantification data and differential expression
results.

\hypertarget{experimental-data}{%
\subsection{Experimental Data}\label{experimental-data}}

The data used in this workflow is available from two packages: the \emph{macrophage}
Bioconductor ExperimentData package and from this workflow package
\emph{fluentGenomics} directly.

The \emph{fluentGenomics} package contains functionality to download and generate a
cached \emph{SummarizedExperiment} object from the normalized ATAC-seq data provided
by \citet{alasooZenodo}. This object contains all 145 ATAC-seq samples across all
experimental conditions as analyzed by \citet{alasoo}. The data can be also be
downloaded directly from the
\href{https://zenodo.org/record/1188300\#.XIAhXlNKjOQ}{Zenodo} deposition.

The following code loads the path to the cached data file, or if it is not
present will create the cache and generate a \emph{SummarizedExperiment} using the
the \emph{BiocFileCache} package \citep{bcfilecache}.

We can then read the cached file and assign it to an object called \texttt{atac}. Note
that this step is not strictly necessary to run the workflow.

\begin{Shaded}
\begin{Highlighting}[]
\KeywordTok{library}\NormalTok{(fluentGenomics)}
\NormalTok{path_to_se <-}\StringTok{ }\KeywordTok{cache_atac_se}\NormalTok{()}
\NormalTok{atac <-}\StringTok{ }\KeywordTok{readRDS}\NormalTok{(path_to_se)}
\end{Highlighting}
\end{Shaded}

A precise description of how we obtained this \emph{SummarizedExperiment} object can
be found in section \ref{atac}.

The \emph{macrophage} package contains RNA-seq quantification from 24 RNA-seq
samples, a subset of the RNA-seq samples generated and analyzed by \citet{alasoo}. The
paired-end reads were quantified using \emph{Salmon} \citep{salmon}, using the Gencode 29
human reference transcripts \citep{gencode}. For more details on quantification, and
the exact code used, consult the vignette of the
\href{http://bioconductor.org/packages/macrophage}{macrophage} package. The package
also contains the \texttt{Snakemake} file that was used to distribute the \emph{Salmon}
quantification jobs on a cluster \citep{snakemake}.

\hypertarget{rna-seq-data-analysis}{%
\section{RNA-seq data analysis}\label{rna-seq-data-analysis}}

\hypertarget{easy-rna-seq-data-import-with-tximeta}{%
\subsection{\texorpdfstring{Easy RNA-seq data import with \emph{tximeta}}{Easy RNA-seq data import with tximeta}}\label{easy-rna-seq-data-import-with-tximeta}}

First, we specify a directory \texttt{dir}, where the quantification files are stored.
You could simply specify this directory with:

\begin{Shaded}
\begin{Highlighting}[]
\NormalTok{dir <-}\StringTok{ "/path/to/quant/files"}
\end{Highlighting}
\end{Shaded}

where the path is relative to your current R session. However, here in this
case we have distributed the files in the \emph{macrophage} package. The relevant
directory can be located and associated files can be located \texttt{system.file}.

\begin{Shaded}
\begin{Highlighting}[]
\NormalTok{dir <-}\StringTok{ }\KeywordTok{system.file}\NormalTok{(}\StringTok{"extdata"}\NormalTok{, }\DataTypeTok{package=}\StringTok{"macrophage"}\NormalTok{)}
\end{Highlighting}
\end{Shaded}

Information about the experiment is contained in the \texttt{coldata.csv} file. We
leverage the \emph{dplyr} and \emph{readr} packages (as part of the \emph{tidyverse}) to read
this file into R \citep{tidyverse}. We will see later that \emph{plyranges} extends these
packages to accommodate genomic ranges.

\begin{Shaded}
\begin{Highlighting}[]
\KeywordTok{library}\NormalTok{(dplyr)}
\end{Highlighting}
\end{Shaded}

\begin{verbatim}
## 
## Attaching package: 'dplyr'
\end{verbatim}

\begin{verbatim}
## The following objects are masked from 'package:stats':
## 
##     filter, lag
\end{verbatim}

\begin{verbatim}
## The following objects are masked from 'package:base':
## 
##     intersect, setdiff, setequal, union
\end{verbatim}

\begin{Shaded}
\begin{Highlighting}[]
\KeywordTok{library}\NormalTok{(readr)}
\NormalTok{colfile <-}\StringTok{ }\KeywordTok{file.path}\NormalTok{(dir, }\StringTok{"coldata.csv"}\NormalTok{)}
\NormalTok{coldata <-}\StringTok{ }\KeywordTok{read_csv}\NormalTok{(colfile) }\OperatorTok
\StringTok{  }\KeywordTok{select}\NormalTok{(}
\NormalTok{    names,}
    \DataTypeTok{id =}\NormalTok{ sample_id,}
    \DataTypeTok{line =}\NormalTok{ line_id,}
    \DataTypeTok{condition =}\NormalTok{ condition_name}
\NormalTok{  ) }\OperatorTok
\StringTok{  }\KeywordTok{mutate}\NormalTok{(}
    \DataTypeTok{files =} \KeywordTok{file.path}\NormalTok{(dir, }\StringTok{"quants"}\NormalTok{, names, }\StringTok{"quant.sf.gz"}\NormalTok{),}
    \DataTypeTok{condition =} \KeywordTok{relevel}\NormalTok{(}\KeywordTok{factor}\NormalTok{(condition), }\StringTok{"naive"}\NormalTok{)}
\NormalTok{  )}
\end{Highlighting}
\end{Shaded}

\begin{verbatim}
## Parsed with column specification:
## cols(
##   names = col_character(),
##   sample_id = col_character(),
##   line_id = col_character(),
##   replicate = col_double(),
##   condition_name = col_character(),
##   macrophage_harvest = col_character(),
##   salmonella_date = col_character(),
##   ng_ul_mean = col_double(),
##   rna_extraction = col_character(),
##   rna_submit = col_character(),
##   library_pool = col_character(),
##   chemistry = col_character(),
##   rna_auto = col_double()
## )
\end{verbatim}

\begin{Shaded}
\begin{Highlighting}[]
\NormalTok{coldata}
\end{Highlighting}
\end{Shaded}

\begin{verbatim}
## # A tibble: 24 x 5
##    names      id     line  condition  files                                     
##    <chr>      <chr>  <chr> <fct>      <chr>                                     
##  1 SAMEA1038… diku_A diku… naive      /Library/Frameworks/R.framework/Versions/…
##  2 SAMEA1038… diku_B diku… IFNg       /Library/Frameworks/R.framework/Versions/…
##  3 SAMEA1038… diku_C diku… SL1344     /Library/Frameworks/R.framework/Versions/…
##  4 SAMEA1038… diku_D diku… IFNg_SL13… /Library/Frameworks/R.framework/Versions/…
##  5 SAMEA1038… eiwy_A eiwy… naive      /Library/Frameworks/R.framework/Versions/…
##  6 SAMEA1038… eiwy_B eiwy… IFNg       /Library/Frameworks/R.framework/Versions/…
##  7 SAMEA1038… eiwy_C eiwy… SL1344     /Library/Frameworks/R.framework/Versions/…
##  8 SAMEA1038… eiwy_D eiwy… IFNg_SL13… /Library/Frameworks/R.framework/Versions/…
##  9 SAMEA1038… fikt_A fikt… naive      /Library/Frameworks/R.framework/Versions/…
## 10 SAMEA1038… fikt_B fikt… IFNg       /Library/Frameworks/R.framework/Versions/…
## # … with 14 more rows
\end{verbatim}

After we have read the \texttt{coldata.csv} file, we select relevant columns from this
table and create a new column called \texttt{files} and transform the existing
\texttt{condition} column to a factor with the ``naive'' cell line as the baseline. The
\texttt{files} column points to the quantifications for each observation - these files
have been gzipped, but would typically not have the `gz' ending if used from
\texttt{salmon} directly. One other thing to note is the use of the pipe
operator,\texttt{\%\textgreater{}\%}, which can be read as then i.e.~first read the data, then select
columns, then mutate them.

Now we have a data frame summarizing the experimental design and the locations
of the quantifications, and are ready for import with \emph{tximeta}. The following
lines of code do a lot of work for the analyst: importing the RNA-seq
quantification (dropping inferential replicates in this case), locating the
relevant reference transcriptome, attaching the transcript ranges to the data,
and fetching genome information. The result is stored in the Bioconductor data
structure called a \emph{SummarizedExperiment}.

\begin{Shaded}
\begin{Highlighting}[]
\KeywordTok{suppressPackageStartupMessages}\NormalTok{(}\KeywordTok{library}\NormalTok{(SummarizedExperiment))}
\KeywordTok{library}\NormalTok{(tximeta)}
\NormalTok{se <-}\StringTok{ }\KeywordTok{tximeta}\NormalTok{(coldata, }\DataTypeTok{dropInfReps=}\OtherTok{TRUE}\NormalTok{)}
\end{Highlighting}
\end{Shaded}

\begin{verbatim}
## importing quantifications
\end{verbatim}

\begin{verbatim}
## reading in files with read_tsv
\end{verbatim}

\begin{verbatim}
## 1 2 3 4 5 6 7 8 9 10 11 12 13 14 15 16 17 18 19 20 21 22 23 24 
## found matching linked transcriptome:
## [ GENCODE - Homo sapiens - release 29 ]
## loading existing TxDb created: 2019-11-22 01:02:58
## Loading required package: GenomicFeatures
## Loading required package: AnnotationDbi
## 
## Attaching package: 'AnnotationDbi'
## 
## The following object is masked from 'package:dplyr':
## 
##     select
## 
## loading existing transcript ranges created: 2019-11-22 01:06:45
## fetching genome info for GENCODE
\end{verbatim}

\begin{Shaded}
\begin{Highlighting}[]
\NormalTok{se}
\end{Highlighting}
\end{Shaded}

\begin{verbatim}
## class: RangedSummarizedExperiment 
## dim: 205870 24 
## metadata(6): tximetaInfo quantInfo ... txomeInfo txdbInfo
## assays(3): counts abundance length
## rownames(205870): ENST00000456328.2 ENST00000450305.2 ...
##   ENST00000387460.2 ENST00000387461.2
## rowData names(3): tx_id gene_id tx_name
## colnames(24): SAMEA103885102 SAMEA103885347 ... SAMEA103885308
##   SAMEA103884949
## colData names(4): names id line condition
\end{verbatim}

The \emph{tximeta} package can be used without an internet connection, in this
case the linked transcriptome can be created directly from a salmon
index and gtf.

\begin{Shaded}
\begin{Highlighting}[]
\KeywordTok{makeLinkedTxome}\NormalTok{(}
  \DataTypeTok{indexDir=}\KeywordTok{file.path}\NormalTok{(dir, }\StringTok{"gencode.v29_salmon_0.12.0"}\NormalTok{),}
  \DataTypeTok{source=}\StringTok{"Gencode"}\NormalTok{,}
  \DataTypeTok{organism=}\StringTok{"Homo sapiens"}\NormalTok{,}
  \DataTypeTok{release=}\StringTok{"29"}\NormalTok{,}
  \DataTypeTok{genome=}\StringTok{"GRCh38"}\NormalTok{,}
  \DataTypeTok{fasta=}\StringTok{"ftp://ftp.ebi.ac.uk/pub/databases/gencode/Gencode_human/release_29/gencode.v29.transcripts.fa.gz"}\NormalTok{,}
  \DataTypeTok{gtf=}\KeywordTok{file.path}\NormalTok{(dir, }\StringTok{"gencode.v29.annotation.gtf.gz"}\NormalTok{), }\CommentTok{# local version}
  \DataTypeTok{write=}\OtherTok{FALSE}
\NormalTok{)}
\end{Highlighting}
\end{Shaded}

Because \emph{tximeta} knows the correct reference transcriptome, we can ask
\emph{tximeta} to summarize the transcript-level data to the gene level using the
methods of \citet{Soneson2015}.

\begin{Shaded}
\begin{Highlighting}[]
\NormalTok{gse <-}\StringTok{ }\KeywordTok{summarizeToGene}\NormalTok{(se)}
\end{Highlighting}
\end{Shaded}

\begin{verbatim}
## loading existing TxDb created: 2019-11-22 01:02:58
\end{verbatim}

\begin{verbatim}
## obtaining transcript-to-gene mapping from TxDb
\end{verbatim}

\begin{verbatim}
## loading existing gene ranges created: 2019-11-23 02:30:13
\end{verbatim}

\begin{verbatim}
## summarizing abundance
\end{verbatim}

\begin{verbatim}
## summarizing counts
\end{verbatim}

\begin{verbatim}
## summarizing length
\end{verbatim}

\hypertarget{preliminary-rna-seq-de-analysis}{%
\subsection{Preliminary RNA-seq DE analysis}\label{preliminary-rna-seq-de-analysis}}

We can easily run a differential expression analysis with \emph{DESeq2} using the
following code chunks \citep{Love2014}. The design indicates that we want to control
for the donor (\texttt{line}) and test for differences in gene expression on the
condition. For a more comprehensive discussion of DE analysis see
\citet{Love2016-f1000} and \citet{Law2018-f1000}.

\begin{Shaded}
\begin{Highlighting}[]
\KeywordTok{library}\NormalTok{(DESeq2)}
\NormalTok{dds <-}\StringTok{ }\KeywordTok{DESeqDataSet}\NormalTok{(gse, }\OperatorTok{~}\NormalTok{line }\OperatorTok{+}\StringTok{ }\NormalTok{condition)}
\end{Highlighting}
\end{Shaded}

\begin{verbatim}
## using counts and average transcript lengths from tximeta
\end{verbatim}

\begin{verbatim}
## Warning in DESeqDataSet(gse, ~line + condition): some variables in design
## formula are characters, converting to factors
\end{verbatim}

\begin{Shaded}
\begin{Highlighting}[]
\CommentTok{# filter out lowly expressed genes}
\CommentTok{# at least 10 counts in at least 6 samples}
\NormalTok{keep <-}\StringTok{ }\KeywordTok{rowSums}\NormalTok{(}\KeywordTok{counts}\NormalTok{(dds) }\OperatorTok{>=}\StringTok{ }\DecValTok{10}\NormalTok{) }\OperatorTok{>=}\StringTok{ }\DecValTok{6}
\NormalTok{dds <-}\StringTok{ }\NormalTok{dds[keep,]}
\end{Highlighting}
\end{Shaded}

Below we set the contrasts on the condition variable, indicating we are
estimating log2 fold changes of IFNg stimulated cell lines against naive cell
lines. We are interested in log fold changes greater than 1 at a false
discovery rate at of 1\%.

\begin{Shaded}
\begin{Highlighting}[]
\NormalTok{dds <-}\StringTok{ }\KeywordTok{DESeq}\NormalTok{(dds)}
\end{Highlighting}
\end{Shaded}

\begin{verbatim}
## estimating size factors
\end{verbatim}

\begin{verbatim}
## using 'avgTxLength' from assays(dds), correcting for library size
\end{verbatim}

\begin{verbatim}
## estimating dispersions
\end{verbatim}

\begin{verbatim}
## gene-wise dispersion estimates
\end{verbatim}

\begin{verbatim}
## mean-dispersion relationship
\end{verbatim}

\begin{verbatim}
## final dispersion estimates
\end{verbatim}

\begin{verbatim}
## fitting model and testing
\end{verbatim}

\begin{Shaded}
\begin{Highlighting}[]
\NormalTok{res <-}\StringTok{ }\KeywordTok{results}\NormalTok{(dds,}
               \DataTypeTok{contrast=}\KeywordTok{c}\NormalTok{(}\StringTok{"condition"}\NormalTok{,}\StringTok{"IFNg"}\NormalTok{,}\StringTok{"naive"}\NormalTok{),}
               \DataTypeTok{lfcThreshold=}\DecValTok{1}\NormalTok{, }\DataTypeTok{alpha=}\FloatTok{0.01}\NormalTok{)}
\end{Highlighting}
\end{Shaded}

To see the results of the expression analysis, we can generate a summary table
and an MA plot:

\begin{Shaded}
\begin{Highlighting}[]
\KeywordTok{summary}\NormalTok{(res)}
\end{Highlighting}
\end{Shaded}

\begin{verbatim}
## 
## out of 17806 with nonzero total read count
## adjusted p-value < 0.01
## LFC > 1.00 (up)    : 502, 2.8%
## LFC < -1.00 (down) : 247, 1.4%
## outliers [1]       : 0, 0%
## low counts [2]     : 0, 0%
## (mean count < 3)
## [1] see 'cooksCutoff' argument of ?results
## [2] see 'independentFiltering' argument of ?results
\end{verbatim}

\begin{Shaded}
\begin{Highlighting}[]
\NormalTok{DESeq2}\OperatorTok{::}\KeywordTok{plotMA}\NormalTok{(res, }\DataTypeTok{ylim=}\KeywordTok{c}\NormalTok{(}\OperatorTok{-}\DecValTok{10}\NormalTok{,}\DecValTok{10}\NormalTok{))}
\end{Highlighting}
\end{Shaded}

\begin{center}\includegraphics{/Users/lee.s/fluentGenomics/docs/fluentGenomics_files/figure-latex/unnamed-chunk-9-1} \end{center}

We now output the results as a \emph{GRanges} object, due to the conventions of
\emph{plyranges} we construct a new column called \texttt{gene\_id} from the row names of
the results. Each row now contains the genomic region (\texttt{seqnames}, \texttt{start},
\texttt{end}, \texttt{strand}) along with corresponding metadata columns (the \texttt{gene\_id} and
the results of the test). Note that \emph{tximeta} has correctly identified the
reference genome as ``hg38'', this has also been annotated onto the target
\emph{GRanges}. This kind of book-keeping is vital once overlap operations are
performed to ensure that \emph{plyranges}

\begin{Shaded}
\begin{Highlighting}[]
\KeywordTok{suppressPackageStartupMessages}\NormalTok{(}\KeywordTok{library}\NormalTok{(plyranges))}
\NormalTok{de_genes <-}\StringTok{ }\KeywordTok{results}\NormalTok{(dds,}
                    \DataTypeTok{contrast=}\KeywordTok{c}\NormalTok{(}\StringTok{"condition"}\NormalTok{,}\StringTok{"IFNg"}\NormalTok{,}\StringTok{"naive"}\NormalTok{),}
                    \DataTypeTok{lfcThreshold=}\DecValTok{1}\NormalTok{,}
                    \DataTypeTok{format=}\StringTok{"GRanges"}\NormalTok{) }\OperatorTok
\StringTok{  }\KeywordTok{names_to_column}\NormalTok{(}\StringTok{"gene_id"}\NormalTok{)}
\NormalTok{de_genes}
\end{Highlighting}
\end{Shaded}

\begin{verbatim}
## GRanges object with 17806 ranges and 7 metadata columns:
##           seqnames              ranges strand |            gene_id
##              <Rle>           <IRanges>  <Rle> |        <character>
##       [1]     chrX 100627109-100639991      - | ENSG00000000003.14
##       [2]    chr20   50934867-50958555      - | ENSG00000000419.12
##       [3]     chr1 169849631-169894267      - | ENSG00000000457.13
##       [4]     chr1 169662007-169854080      + | ENSG00000000460.16
##       [5]     chr1   27612064-27635277      - | ENSG00000000938.12
##       ...      ...                 ...    ... .                ...
##   [17802]    chr10   84167228-84172093      - |  ENSG00000285972.1
##   [17803]     chr6   63572012-63583587      + |  ENSG00000285976.1
##   [17804]    chr16   57177349-57181390      + |  ENSG00000285979.1
##   [17805]     chr8 103398658-103501895      - |  ENSG00000285982.1
##   [17806]    chr10   12563151-12567351      + |  ENSG00000285994.1
##                   baseMean     log2FoldChange              lfcSE
##                  <numeric>          <numeric>          <numeric>
##       [1] 171.570646163445 -0.282245015065582  0.300571026277417
##       [2] 967.751278980391 0.0391222756936352 0.0859707605047955
##       [3] 682.432885098654    1.2846178585311  0.196906721741941
##       [4] 262.963397841117  -1.47187616421189  0.218691645887265
##       [5] 2660.10225731917  0.675478091290521  0.236053041372838
##       ...              ...                ...                ...
##   [17802] 10.0474624496157  0.548451844773876  0.444318686394084
##   [17803] 4586.34616821518 -0.033929582570062  0.188004977365846
##   [17804] 14.2965310090402  0.312347650582085  0.522699844356108
##   [17805] 27.7629588245413  0.994518742790125   1.58237312176743
##   [17806] 6.60408582708505   0.25399752352481    0.5957511892896
##                        stat             pvalue              padj
##                   <numeric>          <numeric>         <numeric>
##       [1]                 0                  1                 1
##       [2]                 0                  1                 1
##       [3]  1.44544511235177  0.148332899695748                 1
##       [4] -2.15772377722715 0.0309493141635637 0.409727500369082
##       [5]                 0                  1                 1
##       ...               ...                ...               ...
##   [17802]                 0                  1                 1
##   [17803]                 0                  1                 1
##   [17804]                 0                  1                 1
##   [17805]                 0                  1                 1
##   [17806]                 0                  1                 1
##   -------
##   seqinfo: 25 sequences (1 circular) from hg38 genome
\end{verbatim}

From this, we can restrict the results where that meet our FDR threshold and
select (and rename) the metadata columns we're interested in:

\begin{Shaded}
\begin{Highlighting}[]
\NormalTok{de_genes <-}\StringTok{ }\NormalTok{de_genes }\OperatorTok
\StringTok{  }\KeywordTok{filter}\NormalTok{(padj }\OperatorTok{<}\StringTok{ }\FloatTok{0.01}\NormalTok{) }\OperatorTok
\StringTok{  }\KeywordTok{select}\NormalTok{(gene_id, }\DataTypeTok{de_log2FC =}\NormalTok{ log2FoldChange, }\DataTypeTok{de_padj =}\NormalTok{ padj)}
\end{Highlighting}
\end{Shaded}

Next we re-run \texttt{results} because we don't want to use an \texttt{lfcThreshold} this
time. This will extract genes which are not differentially expressed according
to the \emph{DESeq2} significance test.

\begin{Shaded}
\begin{Highlighting}[]
\NormalTok{other_genes <-}\StringTok{ }\KeywordTok{results}\NormalTok{(dds,}
                       \DataTypeTok{contrast=}\KeywordTok{c}\NormalTok{(}\StringTok{"condition"}\NormalTok{,}\StringTok{"IFNg"}\NormalTok{,}\StringTok{"naive"}\NormalTok{),}
                       \DataTypeTok{format=}\StringTok{"GRanges"}\NormalTok{) }\OperatorTok
\StringTok{  }\KeywordTok{filter}\NormalTok{(pvalue }\OperatorTok{>}\StringTok{ }\FloatTok{0.1}\NormalTok{) }\OperatorTok
\StringTok{  }\KeywordTok{names_to_column}\NormalTok{(}\StringTok{"gene_id"}\NormalTok{) }\OperatorTok
\StringTok{  }\NormalTok{dplyr}\OperatorTok{::}\KeywordTok{select}\NormalTok{(gene_id,}
                    \DataTypeTok{de_log2FC =}\NormalTok{ log2FoldChange,}
                    \DataTypeTok{de_padj =}\NormalTok{ padj)}
\end{Highlighting}
\end{Shaded}

\hypertarget{atac}{%
\section{ATAC-seq peak differential abundance analysis}\label{atac}}

The following section describes the process we have used for generating a
\emph{GRanges} object of differential peaks from that ATAC-seq data in \citet{alasoo}.

The code chunks for the remainder of this section are not run.

\hypertarget{generating-a-summarizedexperiment-object}{%
\subsection{\texorpdfstring{Generating a \emph{SummarizedExperiment} object}{Generating a SummarizedExperiment object}}\label{generating-a-summarizedexperiment-object}}

The \emph{SummarizedExperiment} object containing ATAC-seq peaks can be created from
the following tab-delimited files found in \citep{alasooZenodo}:

\begin{itemize}
\item
  The annotated peaks: \texttt{ATAC\_peak\_metadata.txt.gz} (5.6M)
\item
  The sample metadata: \texttt{ATAC\_sample\_metadata.txt.gz} (\textless1M)
\item
  The matrix of normalized read counts: \texttt{ATAC\_cqn\_matrix.txt.gz} (109M)
\end{itemize}

First, we read in the peak metadata (locations in the genome), and convert to a
\emph{GRanges} object. The \texttt{as\_granges()} function automatically converts the
data.frame into a \emph{GRanges} object, from that result, we extract the peak\_id
column and set the genome information to the build ``GRCh38''. We know this from
the \href{https://zenodo.org/record/1188300\#.XJOFSlNKiL5}{Zenodo entry}

\begin{Shaded}
\begin{Highlighting}[]
\NormalTok{peaks_df <-}\StringTok{ }\KeywordTok{read_tsv}\NormalTok{(}\StringTok{"ATAC_peak_metadata.txt.gz"}\NormalTok{,}
  \DataTypeTok{col_types =} \KeywordTok{c}\NormalTok{(}\StringTok{"cidciicdc"}\NormalTok{)}
\NormalTok{)}

\NormalTok{peaks_gr <-}\StringTok{ }\NormalTok{peaks_df }\OperatorTok
\StringTok{  }\KeywordTok{as_granges}\NormalTok{(}\DataTypeTok{seqnames =}\NormalTok{ chr) }\OperatorTok
\StringTok{  }\KeywordTok{select}\NormalTok{(}\DataTypeTok{peak_id=}\NormalTok{gene_id) }\OperatorTok
\StringTok{  }\KeywordTok{set_genome_info}\NormalTok{(}\DataTypeTok{genome =} \StringTok{"GRCh38"}\NormalTok{)}
\end{Highlighting}
\end{Shaded}

We also read in the sample metadata:

\begin{Shaded}
\begin{Highlighting}[]
\NormalTok{atac_coldata <-}\StringTok{ }\KeywordTok{read_tsv}\NormalTok{(}\StringTok{"ATAC_sample_metadata.txt.gz"}\NormalTok{) }\OperatorTok
\StringTok{ }\KeywordTok{select}\NormalTok{(}
\NormalTok{    sample_id,}
\NormalTok{    donor,}
    \DataTypeTok{condition =}\NormalTok{ condition_name}
\NormalTok{  ) }\OperatorTok
\StringTok{  }\KeywordTok{mutate}\NormalTok{(}\DataTypeTok{condition =} \KeywordTok{relevel}\NormalTok{(}\KeywordTok{factor}\NormalTok{(condition), }\StringTok{"naive"}\NormalTok{))}
\end{Highlighting}
\end{Shaded}

The ATAC-seq data has already been normalized with \emph{cqn} \citep{Hansen2012} and log2
transformed. Loading the \emph{cqn}-normalized matrix of log2 transformed read
counts takes \textasciitilde30 seconds and loads an object of \textasciitilde370 Mb. We set the column
names so the first row contains the rownames of the matrix, and the remaining
columns are matched to the sample identifiers.

\begin{Shaded}
\begin{Highlighting}[]
\NormalTok{atac_mat <-}\StringTok{ }\KeywordTok{read_tsv}\NormalTok{(}\StringTok{"ATAC_cqn_matrix.txt.gz"}\NormalTok{,}
                     \DataTypeTok{skip =} \DecValTok{1}\NormalTok{,}
                     \DataTypeTok{col_names =}\KeywordTok{c}\NormalTok{(}\StringTok{"rownames"}\NormalTok{, atac_coldata[[}\StringTok{"sample_id"}\NormalTok{]]))}
\NormalTok{rownames <-}\StringTok{ }\NormalTok{atac_mat[[}\StringTok{"rownames"}\NormalTok{]]}
\NormalTok{atac_mat <-}\StringTok{ }\KeywordTok{as.matrix}\NormalTok{(atac_mat[,}\OperatorTok{-}\DecValTok{1}\NormalTok{])}
\KeywordTok{rownames}\NormalTok{(atac_mat) <-}\StringTok{ }\NormalTok{rownames}
\end{Highlighting}
\end{Shaded}

Finally, we then combine the data, and two pieces of metadata into a
\emph{SummarizedExperiment}:

\begin{Shaded}
\begin{Highlighting}[]
\NormalTok{atac <-}\StringTok{ }\KeywordTok{SummarizedExperiment}\NormalTok{(}\KeywordTok{list}\NormalTok{(}\DataTypeTok{cqndata=}\NormalTok{atac_mat),}
                             \DataTypeTok{rowRanges=}\NormalTok{peaks_gr,}
                             \DataTypeTok{colData=}\NormalTok{atac_coldata)}
\end{Highlighting}
\end{Shaded}

\hypertarget{using-limma-for-differential-accessibility}{%
\subsection{Using limma for differential accessibility}\label{using-limma-for-differential-accessibility}}

For assessing differential accessibility, we run \emph{limma} \citep{Smyth2004}, and
generate the a summary of log fold changes and adjusted p-values for the peaks:

\begin{Shaded}
\begin{Highlighting}[]
\KeywordTok{library}\NormalTok{(limma)}
\NormalTok{design <-}\StringTok{ }\KeywordTok{model.matrix}\NormalTok{(}\OperatorTok{~}\NormalTok{donor }\OperatorTok{+}\StringTok{ }\NormalTok{condition, }\KeywordTok{colData}\NormalTok{(atac))}
\NormalTok{fit <-}\StringTok{ }\KeywordTok{lmFit}\NormalTok{(}\KeywordTok{assay}\NormalTok{(atac), design)}
\NormalTok{fit <-}\StringTok{ }\KeywordTok{eBayes}\NormalTok{(fit)}
\NormalTok{idx <-}\StringTok{ }\KeywordTok{which}\NormalTok{(}\KeywordTok{colnames}\NormalTok{(fit}\OperatorTok{$}\NormalTok{coefficients) }\OperatorTok{==}\StringTok{ "conditionIFNg"}\NormalTok{)}
\NormalTok{tt <-}\StringTok{ }\KeywordTok{topTable}\NormalTok{(fit, }\DataTypeTok{coef=}\NormalTok{idx, }\DataTypeTok{sort.by=}\StringTok{"none"}\NormalTok{, }\DataTypeTok{n=}\KeywordTok{nrow}\NormalTok{(atac))}
\end{Highlighting}
\end{Shaded}

We now take the \texttt{rowRanges} of the \emph{SummarizedExperiment} and attach the LFC
and adjusted p-value from \emph{limma}, so that we can consider the overlap with
differential expression. Note that we set the genome build to ``hg38'' and
restyle the chromosome information to use the ``UCSC'' style.

\begin{Shaded}
\begin{Highlighting}[]
\NormalTok{atac_peaks <-}\StringTok{ }\KeywordTok{rowRanges}\NormalTok{(atac) }\OperatorTok
\StringTok{  }\KeywordTok{remove_names}\NormalTok{() }\OperatorTok
\StringTok{  }\KeywordTok{mutate}\NormalTok{(}
    \DataTypeTok{da_log2FC =}\NormalTok{ tt}\OperatorTok{$}\NormalTok{logFC,}
    \DataTypeTok{da_padj =}\NormalTok{ tt}\OperatorTok{$}\NormalTok{adj.P.Val}
\NormalTok{  ) }\OperatorTok
\StringTok{  }\KeywordTok{set_genome_info}\NormalTok{(}\DataTypeTok{genome =} \StringTok{"hg38"}\NormalTok{)}

\KeywordTok{seqlevelsStyle}\NormalTok{(atac_peaks) <-}\StringTok{ "UCSC"}
\end{Highlighting}
\end{Shaded}

\hypertarget{finding-overlaps-with-plyranges}{%
\section{\texorpdfstring{Finding overlaps with \emph{plyranges}}{Finding overlaps with plyranges}}\label{finding-overlaps-with-plyranges}}

We have already used \emph{plyranges} a number of times above, to \texttt{filter}, \texttt{mutate}
and \texttt{select} on \emph{GRanges} objects, as well as ensuring the correct genome
annotation and style has been used.

The final \emph{GRanges} object containing the DA peaks is included in the workflow
package and is loaded below:

\begin{Shaded}
\begin{Highlighting}[]
\KeywordTok{library}\NormalTok{(fluentGenomics)}
\NormalTok{peaks}
\end{Highlighting}
\end{Shaded}

\begin{verbatim}
## GRanges object with 296220 ranges and 3 metadata columns:
##            seqnames              ranges strand |          peak_id
##               <Rle>           <IRanges>  <Rle> |      <character>
##        [1]     chr1          9979-10668      * |      ATAC_peak_1
##        [2]     chr1         10939-11473      * |      ATAC_peak_2
##        [3]     chr1         15505-15729      * |      ATAC_peak_3
##        [4]     chr1         21148-21481      * |      ATAC_peak_4
##        [5]     chr1         21864-22067      * |      ATAC_peak_5
##        ...      ...                 ...    ... .              ...
##   [296216]     chrX 155896572-155896835      * | ATAC_peak_296216
##   [296217]     chrX 155958507-155958646      * | ATAC_peak_296217
##   [296218]     chrX 156016760-156016975      * | ATAC_peak_296218
##   [296219]     chrX 156028551-156029422      * | ATAC_peak_296219
##   [296220]     chrX 156030135-156030785      * | ATAC_peak_296220
##                     da_log2FC              da_padj
##                     <numeric>            <numeric>
##        [1]  0.266185396736073 9.10672732956434e-05
##        [2]   0.32217712436691 2.03434717570469e-05
##        [3] -0.574159538548115 3.41707743345703e-08
##        [4]  -1.14706617895329 8.22298606986521e-26
##        [5] -0.896143162633654 4.79452571676397e-11
##        ...                ...                  ...
##   [296216] -0.834628897017445  1.3354605397165e-11
##   [296217] -0.147537281935847    0.313014754316915
##   [296218] -0.609732301631964 3.62338775135558e-09
##   [296219] -0.347678474957794 6.94823191242968e-06
##   [296220]  0.492442459200901 7.07663984067763e-13
##   -------
##   seqinfo: 23 sequences from hg38 genome; no seqlengths
\end{verbatim}

For the overlap analysis, we filter these to have a nominal FDR bound of 1\$.

\begin{Shaded}
\begin{Highlighting}[]
\NormalTok{da_peaks <-}\StringTok{ }\NormalTok{peaks }\OperatorTok
\StringTok{  }\KeywordTok{filter}\NormalTok{(da_padj }\OperatorTok{<}\StringTok{ }\FloatTok{0.01}\NormalTok{)}
\end{Highlighting}
\end{Shaded}

We now have \emph{GRanges} objects that contain DE genes, genes without strong
signal of DE, and DA peaks. We are ready to perform our original aim and answer
the question: is there an enrichment of DA ATAC-seq peaks in the vicinity of DE
genes?

\hypertarget{down-sampling-non-differentially-expressed-genes}{%
\subsection{Down sampling non-differentially expressed genes}\label{down-sampling-non-differentially-expressed-genes}}

As \emph{plyranges} is built on top of \emph{dplyr} it implements methods for many of
it's verbs for \emph{GRanges} objects. Here we can use, \texttt{slice} to randomly sample
the rows of the \texttt{other\_genes}. The \texttt{sample.int} function will generate random
samples of size equal to the number of DE-genes from the number of rows in
\texttt{other\_genes}:

\begin{Shaded}
\begin{Highlighting}[]
\NormalTok{size <-}\StringTok{ }\KeywordTok{length}\NormalTok{(de_genes)}
\KeywordTok{slice}\NormalTok{(other_genes, }\KeywordTok{sample.int}\NormalTok{(}\KeywordTok{n}\NormalTok{(), size))}
\end{Highlighting}
\end{Shaded}

\begin{verbatim}
## GRanges object with 749 ranges and 3 metadata columns:
##         seqnames              ranges strand |            gene_id
##            <Rle>           <IRanges>  <Rle> |        <character>
##     [1]     chr6   96521595-96555276      + |  ENSG00000014123.9
##     [2]    chr12   66188879-66254622      + | ENSG00000090376.10
##     [3]     chr3   44729596-44737083      + | ENSG00000186446.11
##     [4]     chr7 157138913-157269372      + | ENSG00000009335.17
##     [5]     chr7     6577434-6589374      + | ENSG00000136247.14
##     ...      ...                 ...    ... .                ...
##   [745]    chr16   30378106-30400108      + | ENSG00000180035.12
##   [746]    chr19   19865886-19894674      + |  ENSG00000256771.3
##   [747]     chrX 119871487-119876662      + |  ENSG00000125356.6
##   [748]     chr2 113705011-113756823      - | ENSG00000115084.13
##   [749]    chr11 119206276-119313926      + |  ENSG00000110395.6
##                   de_log2FC           de_padj
##                   <numeric>         <numeric>
##     [1]   0.157288238952329 0.393474621468054
##     [2]   0.192675255367846 0.638575038094171
##     [3]   0.294974004663957 0.677850298147425
##     [4]  0.0541851010739642 0.688618637431238
##     [5]  -0.168092883269452 0.412214659609548
##     ...                 ...               ...
##   [745]  -0.253169808554397 0.437374379292198
##   [746]  -0.288346936786858 0.321576385838034
##   [747] -0.0656250155572518 0.660307119203159
##   [748]  0.0920937558644995  0.78240487118216
##   [749]   0.111941856478517 0.649460473096424
##   -------
##   seqinfo: 25 sequences (1 circular) from hg38 genome
\end{verbatim}

We can repeat this many times to create many samples via replicate:

\begin{Shaded}
\begin{Highlighting}[]
\CommentTok{# set a seed for the results}
\KeywordTok{set.seed}\NormalTok{(}\DecValTok{2019-08-02}\NormalTok{)}
\NormalTok{boot_genes <-}\StringTok{ }\KeywordTok{replicate}\NormalTok{(}\DecValTok{10}\NormalTok{,}
                        \KeywordTok{slice}\NormalTok{(other_genes, }\KeywordTok{sample.int}\NormalTok{(}\KeywordTok{n}\NormalTok{(), size)),}
                        \DataTypeTok{simplify =} \OtherTok{FALSE}\NormalTok{)}
\end{Highlighting}
\end{Shaded}

This creates a list of \emph{GRanges} objects as a list, we can bind these together
using the \texttt{bind\_ranges} function. This function creates a new column called
``resample'' on the result that identifies each of the input \emph{GRanges} objects:

\begin{Shaded}
\begin{Highlighting}[]
\NormalTok{boot_genes <-}\StringTok{ }\KeywordTok{bind_ranges}\NormalTok{(boot_genes, }\DataTypeTok{.id =} \StringTok{"resample"}\NormalTok{)}
\end{Highlighting}
\end{Shaded}

Similarly, we can then combine the \texttt{boot\_genes} \emph{GRanges}, with the DE
\emph{GRanges} object. As the resample column was not present on the DE \emph{GRanges}
object, this is given a missing value which we recode to a 0 using \texttt{mutate()}

\begin{Shaded}
\begin{Highlighting}[]
\NormalTok{all_genes <-}\StringTok{ }\KeywordTok{bind_ranges}\NormalTok{(}
  \DataTypeTok{de=}\NormalTok{de_genes,}
  \DataTypeTok{not_de =}\NormalTok{ boot_genes,}
  \DataTypeTok{.id=}\StringTok{"origin"}
\NormalTok{  ) }\OperatorTok
\StringTok{  }\KeywordTok{mutate}\NormalTok{(}
    \DataTypeTok{origin =} \KeywordTok{factor}\NormalTok{(origin, }\KeywordTok{c}\NormalTok{(}\StringTok{"de"}\NormalTok{, }\StringTok{"not_de"}\NormalTok{)),}
    \DataTypeTok{resample =} \KeywordTok{ifelse}\NormalTok{(}\KeywordTok{is.na}\NormalTok{(resample), 0L, }\KeywordTok{as.integer}\NormalTok{(resample))}
\NormalTok{  )}
\NormalTok{all_genes}
\end{Highlighting}
\end{Shaded}

\begin{verbatim}
## GRanges object with 8239 ranges and 5 metadata columns:
##          seqnames              ranges strand |            gene_id
##             <Rle>           <IRanges>  <Rle> |        <character>
##      [1]     chr1 196651878-196747504      + | ENSG00000000971.15
##      [2]     chr6   46129993-46146699      + |  ENSG00000001561.6
##      [3]     chr4   17577192-17607972      + | ENSG00000002549.12
##      [4]     chr7 150800403-150805120      + |  ENSG00000002933.8
##      [5]     chr4   15778275-15853230      + | ENSG00000004468.12
##      ...      ...                 ...    ... .                ...
##   [8235]     chr3   72749277-72861914      - | ENSG00000144736.13
##   [8236]    chr17   29566052-29573157      + | ENSG00000167543.15
##   [8237]     chr7 129225023-129430211      + | ENSG00000158467.16
##   [8238]     chr2   24029340-24049575      - | ENSG00000163026.11
##   [8239]    chr16     1826941-1840207      + | ENSG00000180185.11
##                   de_log2FC              de_padj  resample   origin
##                   <numeric>            <numeric> <integer> <factor>
##      [1]   4.98711071930695 1.37057050625117e-13         0       de
##      [2]   1.92721595378787  3.1747750217733e-05         0       de
##      [3]   2.93372501059128  2.0131038573066e-11         0       de
##      [4]   3.16721751137972 1.07359906028984e-08         0       de
##      [5]   5.40894352968188 4.82904694023763e-18         0       de
##      ...                ...                  ...       ...      ...
##   [8235] -0.324057975853188    0.531385106367204        10   not_de
##   [8236] 0.0582048743660616    0.681056515870893        10   not_de
##   [8237]  0.284556421479042    0.378275048382877        10   not_de
##   [8238] -0.130576070173704    0.601621121971093        10   not_de
##   [8239]  0.173903843804872    0.568644093307946        10   not_de
##   -------
##   seqinfo: 25 sequences (1 circular) from hg38 genome
\end{verbatim}

\hypertarget{expanding-genomic-coordinates-around-the-transcription-start-site}{%
\subsection{Expanding genomic coordinates around the transcription start site}\label{expanding-genomic-coordinates-around-the-transcription-start-site}}

Now we would like to modify our gene ranges so their width is 10 kilobases on
either side of their transcription start site (TSS). There are many ways one
could do this but we prefer an approach via the anchoring methods in
\emph{plyranges}. Because there is a mutual dependence between the start, end, width
and strand of a \emph{GRanges} object, we define anchors to fix one of start and
end, while modifying the width. As an example to extract just the TSS, we can
anchor by the 5' end of the range and modify the width of the range to equal 1.

\begin{Shaded}
\begin{Highlighting}[]
\NormalTok{all_genes <-}\StringTok{ }\NormalTok{all_genes }\OperatorTok
\StringTok{  }\KeywordTok{anchor_5p}\NormalTok{() }\OperatorTok
\StringTok{  }\KeywordTok{mutate}\NormalTok{(}\DataTypeTok{width =} \DecValTok{1}\NormalTok{)}
\end{Highlighting}
\end{Shaded}

Anchoring by the 5' end of a range will fix the end of negatively stranded
ranges, and fix the start of positively stranded ranges.

We can then repeat the same pattern but this time using \texttt{anchor\_center()} to
tell \emph{plyranges} that we are making the TSS the midpoint of a range that has
total width of 20kb or 10kb both upstream and downstream of the TSS.

\begin{Shaded}
\begin{Highlighting}[]
\NormalTok{all_genes <-all_genes }\OperatorTok
\StringTok{  }\KeywordTok{anchor_center}\NormalTok{() }\OperatorTok
\StringTok{  }\KeywordTok{mutate}\NormalTok{(}\DataTypeTok{width=}\DecValTok{2}\OperatorTok{*}\FloatTok{1e4}\NormalTok{)}
\end{Highlighting}
\end{Shaded}

\hypertarget{use-overlap-joins-to-find-relative-enrichment}{%
\subsection{Use overlap joins to find relative enrichment}\label{use-overlap-joins-to-find-relative-enrichment}}

We are now ready to compute overlaps between RNA-seq genes (our DE set and
bootstrap samples) and the ATAC-seq peaks. In \emph{plyranges}, overlaps are defined
as joins between two \emph{GRanges} objects: a \emph{left} and a \emph{right} \emph{GRanges}
object. In an overlap join, a match is any range on the \emph{left} \emph{GRanges} that
is overlapped by the \emph{right} \emph{GRanges}. One powerful aspect of the overlap
joins is that the result maintains all (metadata) columns from each of the
\emph{left} and \emph{right} ranges which makes downstream summaries easy to compute.

To combine the DE genes with the DA peaks, we perform a left overlap join. This
returns to us the \texttt{all\_genes} ranges (potentially with duplication), but with
the metadata columns from those overlapping DA peaks. For any gene that has no
overlaps, the DA peak columns will have \texttt{NA}'s.

\begin{Shaded}
\begin{Highlighting}[]
\NormalTok{overlap_genes <-}\StringTok{ }\NormalTok{all_genes }\OperatorTok
\StringTok{  }\KeywordTok{join_overlap_left}\NormalTok{(da_peaks)}
\NormalTok{overlap_genes}
\end{Highlighting}
\end{Shaded}

\begin{verbatim}
## GRanges object with 27591 ranges and 8 metadata columns:
##           seqnames              ranges strand |            gene_id
##              <Rle>           <IRanges>  <Rle> |        <character>
##       [1]     chr1 196641878-196661877      + | ENSG00000000971.15
##       [2]     chr6   46119993-46139992      + |  ENSG00000001561.6
##       [3]     chr4   17567192-17587191      + | ENSG00000002549.12
##       [4]     chr4   17567192-17587191      + | ENSG00000002549.12
##       [5]     chr4   17567192-17587191      + | ENSG00000002549.12
##       ...      ...                 ...    ... .                ...
##   [27587]    chr17   29556052-29576051      + | ENSG00000167543.15
##   [27588]     chr7 129215023-129235022      + | ENSG00000158467.16
##   [27589]     chr2   24039575-24059574      - | ENSG00000163026.11
##   [27590]    chr16     1816941-1836940      + | ENSG00000180185.11
##   [27591]    chr16     1816941-1836940      + | ENSG00000180185.11
##                    de_log2FC              de_padj  resample   origin
##                    <numeric>            <numeric> <integer> <factor>
##       [1]   4.98711071930695 1.37057050625117e-13         0       de
##       [2]   1.92721595378787  3.1747750217733e-05         0       de
##       [3]   2.93372501059128  2.0131038573066e-11         0       de
##       [4]   2.93372501059128  2.0131038573066e-11         0       de
##       [5]   2.93372501059128  2.0131038573066e-11         0       de
##       ...                ...                  ...       ...      ...
##   [27587] 0.0582048743660616    0.681056515870893        10   not_de
##   [27588]  0.284556421479042    0.378275048382877        10   not_de
##   [27589] -0.130576070173704    0.601621121971093        10   not_de
##   [27590]  0.173903843804872    0.568644093307946        10   not_de
##   [27591]  0.173903843804872    0.568644093307946        10   not_de
##                    peak_id          da_log2FC              da_padj
##                <character>          <numeric>            <numeric>
##       [1]  ATAC_peak_21236 -0.546582189082724 0.000115273676444232
##       [2] ATAC_peak_231183   1.45329684862127  9.7322474682763e-17
##       [3] ATAC_peak_193578  0.222371496904895 3.00939005719989e-11
##       [4] ATAC_peak_193579 -0.281615137872819 7.99888515457195e-05
##       [5] ATAC_peak_193580  0.673705317951604 7.60042918890061e-15
##       ...              ...                ...                  ...
##   [27587] ATAC_peak_109304  0.211750928531232  0.00111289505290053
##   [27588] ATAC_peak_255700  0.177364068655037 5.25384772617076e-09
##   [27589] ATAC_peak_133247 -0.266265981992122  1.0155043266685e-07
##   [27590]  ATAC_peak_97065 -0.405271356218346 1.28054921946532e-05
##   [27591]  ATAC_peak_97067  0.289105317951603 5.87369828076092e-07
##   -------
##   seqinfo: 25 sequences (1 circular) from hg38 genome
\end{verbatim}

Now we can ask, how many DA peaks are near DE genes relative to ``other'' non-DE
genes? A gene may appear more than once, since multiple peaks may overlap a
single gene or because we have re-sampled the same gene more than once.

For each gene (that is the combination of chromosome, the start, end and
strand), and the ``origin'' (DE vs not-DE) we can compute the distinct number of
peaks for each gene and the maximum peak based on log FC. This is achieved via
\texttt{reduce\_ranges\_directed}, which allows an aggregation to result in a GRanges
object via merging neighboring genomic regions. The use of the directed suffix
indicates we're maintaining strand information. In this case, we are simply
merging ranges via the groups we mentioned above. We also have to account for
the number of resamples we have performed when counting if there are any peaks,
to ensure we do not double count the same peak:

\begin{Shaded}
\begin{Highlighting}[]
\NormalTok{any_peaks <-}\StringTok{ }\NormalTok{overlap_genes }\OperatorTok
\StringTok{  }\KeywordTok{group_by}\NormalTok{(gene_id, origin)  }\OperatorTok
\StringTok{  }\KeywordTok{reduce_ranges_directed}\NormalTok{(}
    \DataTypeTok{any =} \KeywordTok{sum}\NormalTok{(}\OperatorTok{!}\KeywordTok{is.na}\NormalTok{(da_padj)) }\OperatorTok{/}\StringTok{ }\KeywordTok{n_distinct}\NormalTok{(resample),}
    \DataTypeTok{max_logFC =} \KeywordTok{max}\NormalTok{(}\KeywordTok{abs}\NormalTok{(da_log2FC))}
\NormalTok{  )}
\end{Highlighting}
\end{Shaded}

We can then filter genes if there have any peaks and compare the peak fold
changes between non-DE and DE genes using a boxplot:

\begin{Shaded}
\begin{Highlighting}[]
\KeywordTok{library}\NormalTok{(ggplot2)}
\NormalTok{any_peaks }\OperatorTok
\StringTok{  }\KeywordTok{filter}\NormalTok{(any }\OperatorTok{>}\StringTok{ }\DecValTok{0}\NormalTok{) }\OperatorTok
\StringTok{  }\KeywordTok{as.data.frame}\NormalTok{() }\OperatorTok
\StringTok{  }\KeywordTok{ggplot}\NormalTok{(}\KeywordTok{aes}\NormalTok{(origin, max_logFC)) }\OperatorTok{+}
\StringTok{  }\KeywordTok{geom_boxplot}\NormalTok{()}
\end{Highlighting}
\end{Shaded}

\begin{center}\includegraphics{/Users/lee.s/fluentGenomics/docs/fluentGenomics_files/figure-latex/unnamed-chunk-29-1} \end{center}

In general, the DE genes have larger DA fold changes relative to the non-DE
genes.

Next we examine how changes in DA LFC alter enrichment for DE genes. First, we
want to know how the number of peaks within DE genes and non-DE genes change as
we change threshold values on the peak LFC. As an example, we could compute
this by arbitrarily chosen LFC thresholds as follows:

\begin{Shaded}
\begin{Highlighting}[]
\NormalTok{overlap_tab <-}\StringTok{ }\NormalTok{overlap_genes }\OperatorTok
\StringTok{  }\KeywordTok{group_by}\NormalTok{(origin) }\OperatorTok
\StringTok{  }\KeywordTok{summarize}\NormalTok{(}\DataTypeTok{any=}\KeywordTok{sum}\NormalTok{(}\OperatorTok{!}\KeywordTok{is.na}\NormalTok{(da_padj)) }\OperatorTok{/}\StringTok{ }\KeywordTok{n_distinct}\NormalTok{(resample),}
            \DataTypeTok{lfc1 =}\KeywordTok{sum}\NormalTok{(}\KeywordTok{abs}\NormalTok{(da_log2FC) }\OperatorTok{>}\StringTok{ }\DecValTok{1}\NormalTok{, }\DataTypeTok{na.rm=}\OtherTok{TRUE}\NormalTok{)}\OperatorTok{/}\StringTok{ }\KeywordTok{n_distinct}\NormalTok{(resample),}
            \DataTypeTok{lfc2=} \KeywordTok{sum}\NormalTok{(}\KeywordTok{abs}\NormalTok{(da_log2FC) }\OperatorTok{>}\StringTok{ }\DecValTok{2}\NormalTok{, }\DataTypeTok{na.rm=}\OtherTok{TRUE}\NormalTok{)}\OperatorTok{/}\StringTok{ }\KeywordTok{n_distinct}\NormalTok{(resample))}
\NormalTok{overlap_tab}
\end{Highlighting}
\end{Shaded}

\begin{verbatim}
## DataFrame with 2 rows and 4 columns
##     origin       any      lfc1      lfc2
##   <factor> <numeric> <numeric> <numeric>
## 1       de      3416      1097       234
## 2   not_de    2362.3     443.8      32.3
\end{verbatim}

Then using \texttt{summarize\_all()} from \emph{dplyr} we divide the rows of the above
computation, to see that the relative enrichment increases for a `larger' LFC:

\begin{Shaded}
\begin{Highlighting}[]
\NormalTok{overlap_tab }\OperatorTok
\StringTok{  }\KeywordTok{as.data.frame}\NormalTok{() }\OperatorTok
\StringTok{  }\KeywordTok{select}\NormalTok{(}\OperatorTok{-}\NormalTok{origin) }\OperatorTok
\StringTok{  }\KeywordTok{summarize_all}\NormalTok{(}\DataTypeTok{.funs =} \OperatorTok{~}\KeywordTok{Reduce}\NormalTok{(}\StringTok{"/"}\NormalTok{, .))}
\end{Highlighting}
\end{Shaded}

\begin{verbatim}
##        any     lfc1     lfc2
## 1 1.446048 2.471834 7.244582
\end{verbatim}

Due to the one-to-many mappings of DE genes to peaks, it is unknown if we have
the same number of DE genes participating or less, as we increase the threshold
on the peak LFC. This can be accounted for by grouping and aggregating twice.
First, the number of peaks that meet the thresholds are computed within each
gene, origin and resample group. Second, within the origin column, we compute
the total number of peaks that meet the target threshold and the number of
genes that have more than zero peaks (again averaging over the number of
resamples).

\begin{Shaded}
\begin{Highlighting}[]
\NormalTok{overlap_genes }\OperatorTok
\StringTok{  }\KeywordTok{group_by}\NormalTok{(gene_id, origin, resample) }\OperatorTok
\StringTok{  }\KeywordTok{reduce_ranges_directed}\NormalTok{(}
    \DataTypeTok{lfc1 =}\KeywordTok{sum}\NormalTok{(}\KeywordTok{abs}\NormalTok{(da_log2FC) }\OperatorTok{>}\StringTok{ }\DecValTok{1}\NormalTok{, }\DataTypeTok{na.rm=}\OtherTok{TRUE}\NormalTok{),}
    \DataTypeTok{lfc2=} \KeywordTok{sum}\NormalTok{(}\KeywordTok{abs}\NormalTok{(da_log2FC) }\OperatorTok{>}\StringTok{ }\DecValTok{2}\NormalTok{, }\DataTypeTok{na.rm=}\OtherTok{TRUE}\NormalTok{)}
\NormalTok{  ) }\OperatorTok
\StringTok{  }\KeywordTok{group_by}\NormalTok{(origin) }\OperatorTok
\StringTok{  }\KeywordTok{summarize}\NormalTok{(}
    \DataTypeTok{lfc1_gene_count =} \KeywordTok{sum}\NormalTok{(lfc1 }\OperatorTok{>}\StringTok{ }\DecValTok{0}\NormalTok{) }\OperatorTok{/}\StringTok{ }\KeywordTok{n_distinct}\NormalTok{(resample),}
    \DataTypeTok{lfc1_peak_count =} \KeywordTok{sum}\NormalTok{(lfc1) }\OperatorTok{/}\StringTok{ }\KeywordTok{n_distinct}\NormalTok{(resample),}
    \DataTypeTok{lfc2_gene_count =} \KeywordTok{sum}\NormalTok{(lfc2 }\OperatorTok{>}\StringTok{ }\DecValTok{0}\NormalTok{) }\OperatorTok{/}\StringTok{ }\KeywordTok{n_distinct}\NormalTok{(resample),}
    \DataTypeTok{lfc2_peak_count =} \KeywordTok{sum}\NormalTok{(lfc2) }\OperatorTok{/}\StringTok{ }\KeywordTok{n_distinct}\NormalTok{(resample)}
\NormalTok{  )}
\end{Highlighting}
\end{Shaded}

\begin{verbatim}
## DataFrame with 2 rows and 5 columns
##     origin lfc1_gene_count lfc1_peak_count lfc2_gene_count lfc2_peak_count
##   <factor>       <numeric>       <numeric>       <numeric>       <numeric>
## 1       de             515            1097             151             234
## 2   not_de           301.4           443.8            30.4            32.3
\end{verbatim}

To do this for many thresholds is cumbersome and would create a lot of
duplicate code, instead we create a single function called
\texttt{count\_above\_threshold} that accepts a GRanges object, and computes the peak
count that meets the threshold.

\begin{Shaded}
\begin{Highlighting}[]
\NormalTok{count_above_threshold <-}\StringTok{ }\ControlFlowTok{function}\NormalTok{(.data, threshold) \{}
  \KeywordTok{reduce_ranges_directed}\NormalTok{(.data,}
                         \DataTypeTok{threshold =}\NormalTok{ threshold,}
                         \DataTypeTok{value =} \KeywordTok{sum}\NormalTok{(}\KeywordTok{abs}\NormalTok{(da_log2FC) }\OperatorTok{>}\StringTok{ }\NormalTok{threshold, }\DataTypeTok{na.rm =} \OtherTok{TRUE}\NormalTok{)}
\NormalTok{  )}
\NormalTok{\}}
\end{Highlighting}
\end{Shaded}

The above function will compute the counts for any arbitrary threshold, now we
need to apply it over possible LFC thresholds of interest. We choose a grid of
one hundred thresholds based on the range of absolute LFC values in the
\texttt{da\_peaks} \emph{GRanges} object:

\begin{Shaded}
\begin{Highlighting}[]
\NormalTok{thresholds <-}\StringTok{ }\NormalTok{da_peaks }\OperatorTok
\StringTok{  }\KeywordTok{mutate}\NormalTok{(}\DataTypeTok{abs_lfc =} \KeywordTok{abs}\NormalTok{(da_log2FC)) }\OperatorTok
\StringTok{  }\KeywordTok{with}\NormalTok{(}
    \KeywordTok{seq}\NormalTok{(}\KeywordTok{min}\NormalTok{(abs_lfc), }\KeywordTok{max}\NormalTok{(abs_lfc), }\DataTypeTok{length.out =} \DecValTok{100}\NormalTok{)}
\NormalTok{  )}
\end{Highlighting}
\end{Shaded}

The peaks are computed for each value by applying \texttt{count\_above\_threshold()} on
the grouped \emph{GRanges} object for each threshold and then binding the result.
This could be done easily in parallel with \texttt{BiocParallel} but here we compute
everything serially.

\begin{Shaded}
\begin{Highlighting}[]
\NormalTok{by_gene_origin <-}\StringTok{ }\NormalTok{overlap_genes }\OperatorTok
\StringTok{  }\KeywordTok{group_by}\NormalTok{(gene_id, origin, resample)}

\NormalTok{all_thresholds <-}\StringTok{ }\KeywordTok{bind_ranges}\NormalTok{(}
  \KeywordTok{lapply}\NormalTok{(thresholds, count_above_threshold, }\DataTypeTok{.data =}\NormalTok{ by_gene_origin)}
\NormalTok{)}
\end{Highlighting}
\end{Shaded}

This creates a very \emph{long} GRanges object. To compute the peak and gene counts
for each threshold, we apply the same summarization as before:

\begin{Shaded}
\begin{Highlighting}[]
\NormalTok{counts_by_threshold <-}\StringTok{ }\NormalTok{all_thresholds }\OperatorTok
\StringTok{  }\KeywordTok{group_by}\NormalTok{(origin, threshold) }\OperatorTok
\StringTok{  }\KeywordTok{summarize}\NormalTok{(}
    \DataTypeTok{gene_count =} \KeywordTok{sum}\NormalTok{(value }\OperatorTok{>}\StringTok{ }\DecValTok{0}\NormalTok{) }\OperatorTok{/}\StringTok{ }\KeywordTok{n_distinct}\NormalTok{(resample),}
    \DataTypeTok{peak_count =} \KeywordTok{sum}\NormalTok{(value) }\OperatorTok{/}\StringTok{ }\KeywordTok{n_distinct}\NormalTok{(resample)}
\NormalTok{  )}
\NormalTok{counts_by_threshold}
\end{Highlighting}
\end{Shaded}

\begin{verbatim}
## DataFrame with 200 rows and 4 columns
##       origin          threshold gene_count peak_count
##     <factor>          <numeric>  <numeric>  <numeric>
## 1         de 0.0658243106359027        727       3416
## 2         de  0.118483961449043        727       3392
## 3         de  0.171143612262182        727       3304
## 4         de  0.223803263075322        724       3160
## 5         de  0.276462913888462        723       3045
## ...      ...                ...        ...        ...
## 196   not_de   5.06849113788419          0          0
## 197   not_de   5.12115078869733          0          0
## 198   not_de   5.17381043951047          0          0
## 199   not_de   5.22647009032361          0          0
## 200   not_de   5.27912974113675          0          0
\end{verbatim}

Again we can compute the relative enrichment in LFCs in the same manner as
before and visualize how enrichment changes as the threshold value increases:

\begin{Shaded}
\begin{Highlighting}[]
\NormalTok{enrichment <-}\StringTok{ }\NormalTok{counts_by_threshold }\OperatorTok
\StringTok{  }\KeywordTok{as.data.frame}\NormalTok{() }\OperatorTok
\StringTok{  }\KeywordTok{group_by}\NormalTok{(threshold) }\OperatorTok
\StringTok{  }\KeywordTok{summarize}\NormalTok{(}\DataTypeTok{enrichment =} \KeywordTok{Reduce}\NormalTok{(}\StringTok{"/"}\NormalTok{, peak_count))}

\NormalTok{enrichment }\OperatorTok
\StringTok{  }\KeywordTok{ggplot}\NormalTok{(}\KeywordTok{aes}\NormalTok{(}\DataTypeTok{x =}\NormalTok{ threshold, }\DataTypeTok{y =}\NormalTok{ enrichment)) }\OperatorTok{+}
\StringTok{  }\KeywordTok{geom_line}\NormalTok{() }\OperatorTok{+}
\StringTok{  }\KeywordTok{labs}\NormalTok{(}\DataTypeTok{x =} \StringTok{"logFC threshold"}\NormalTok{, }\DataTypeTok{y =} \StringTok{"Relative Enrichment"}\NormalTok{)}
\end{Highlighting}
\end{Shaded}

\begin{verbatim}
## Warning: Removed 4 rows containing missing values (geom_path).
\end{verbatim}

\begin{center}\includegraphics{/Users/lee.s/fluentGenomics/docs/fluentGenomics_files/figure-latex/unnamed-chunk-37-1} \end{center}

We computed the sum of DA peaks near the DE genes, for increasing LFC
thresholds on the accessibility change. As we increased the threshold, the
number of total peaks went down (likewise the mean number of DA peaks per
gene). It is also likely the number of DE genes with a DA peak nearby with such
a large change went down - we can check this by plotting the number of DE genes
against the number of DA peaks:

\begin{Shaded}
\begin{Highlighting}[]
\NormalTok{counts_by_threshold }\OperatorTok
\StringTok{  }\KeywordTok{as.data.frame}\NormalTok{() }\OperatorTok
\StringTok{  }\KeywordTok{ggplot}\NormalTok{(}\KeywordTok{aes}\NormalTok{(}\DataTypeTok{x =}\NormalTok{ gene_count,}
             \DataTypeTok{y =}\NormalTok{ peak_count,}
             \DataTypeTok{color =}\NormalTok{ threshold)) }\OperatorTok{+}
\StringTok{  }\KeywordTok{geom_point}\NormalTok{() }\OperatorTok{+}
\StringTok{  }\KeywordTok{scale_color_viridis_c}\NormalTok{() }\OperatorTok{+}
\StringTok{  }\KeywordTok{facet_wrap}\NormalTok{(}\OperatorTok{~}\StringTok{ }\NormalTok{origin) }\OperatorTok{+}
\StringTok{  }\KeywordTok{labs}\NormalTok{(}\DataTypeTok{x =} \StringTok{"Number of Genes"}\NormalTok{,}
       \DataTypeTok{y =} \StringTok{"Number of Peaks"}\NormalTok{,}
       \DataTypeTok{color =} \StringTok{"LFC threshold"}\NormalTok{)}
\end{Highlighting}
\end{Shaded}

\begin{center}\includegraphics{/Users/lee.s/fluentGenomics/docs/fluentGenomics_files/figure-latex/unnamed-chunk-38-1} \end{center}

\hypertarget{discussion}{%
\section{Discussion}\label{discussion}}

We have shown that using \emph{plyranges} and \emph{tximeta} (with support of
Bioconductor and \emph{tidyverse} packages) that we can fluently iterate through the
biological data science workflow: from import, through to modeling, wrangling
and visualization.

Using \emph{tximeta}, we have shown that it is straightforward to import RNA-seq
quantification data, and that by ensuring the proper metadata is associated
with it, we can guard against any mistakes in downstream analyses.

Using \emph{plyranges}, we have extended the principles of the \emph{tidyverse} to
genomic ranges, and that by design we can leverage those packages to understand
data measured along the genome. We have shown that analyses performed with
\emph{plyranges} clearly and (relatively) concisely express their intent; in most
cases the code we have written closely matches it's description in English and
clarifies how the features of a genomic range is being modified.

There are several further steps that would be interesting to perform in this
analysis; for example, we could modify window size around the TSS to see how it
effects enrichment and vary the cut-offs applied to FDR percentages applied to
both the DE and DA peaks.

\hypertarget{software-availability}{%
\section{Software Availability}\label{software-availability}}

The workflow materials, including this article can be fully reproduced
following the instructions found at the Github repository
\href{https://github.com/sa-lee/fluentGenomics}{sa-lee/fluentGenomics}. Moreover,
the workflow and all downstream dependencies can be installed

This article and the analyses were performed with R \citep{baser} using the
\emph{rmarkdown} \citep{rmarkdown}, \emph{knitr} \citep{knitr, xie2015} and \emph{BiocWorkflowTools}
\citep{bcworkflowtools} packages.

\hypertarget{session-info}{%
\subsection{Session Info}\label{session-info}}

\begin{Shaded}
\begin{Highlighting}[]
\KeywordTok{sessionInfo}\NormalTok{()}
\end{Highlighting}
\end{Shaded}

\begin{verbatim}
## R version 3.6.1 (2019-07-05)
## Platform: x86_64-apple-darwin15.6.0 (64-bit)
## Running under: macOS Mojave 10.14.6
## 
## Matrix products: default
## BLAS:   /System/Library/Frameworks/Accelerate.framework/Versions/A/Frameworks/vecLib.framework/Versions/A/libBLAS.dylib
## LAPACK: /Library/Frameworks/R.framework/Versions/3.6/Resources/lib/libRlapack.dylib
## 
## locale:
## [1] en_AU.UTF-8/en_AU.UTF-8/en_AU.UTF-8/C/en_AU.UTF-8/en_AU.UTF-8
## 
## attached base packages:
## [1] parallel  stats4    stats     graphics  grDevices utils     datasets 
## [8] methods   base     
## 
## other attached packages:
##  [1] ggplot2_3.2.1.9000          plyranges_1.7.6            
##  [3] DESeq2_1.26.0               GenomicFeatures_1.38.0     
##  [5] AnnotationDbi_1.48.0        SummarizedExperiment_1.16.0
##  [7] DelayedArray_0.12.0         BiocParallel_1.20.0        
##  [9] matrixStats_0.55.0          Biobase_2.46.0             
## [11] GenomicRanges_1.38.0        GenomeInfoDb_1.22.0        
## [13] IRanges_2.20.1              S4Vectors_0.24.1           
## [15] BiocGenerics_0.32.0         readr_1.3.1                
## [17] dplyr_0.8.3                 tximeta_1.4.2              
## [19] fluentGenomics_0.0.3       
## 
## loaded via a namespace (and not attached):
##   [1] colorspace_1.4-1         htmlTable_1.13.3         XVector_0.26.0          
##   [4] base64enc_0.1-3          fs_1.3.1                 rstudioapi_0.10         
##   [7] farver_2.0.1             bit64_0.9-7              fansi_0.4.0             
##  [10] xml2_1.2.2               splines_3.6.1            tximport_1.14.0         
##  [13] geneplotter_1.64.0       knitr_1.26               zeallot_0.1.0           
##  [16] Formula_1.2-3            jsonlite_1.6             Rsamtools_2.2.1         
##  [19] annotate_1.64.0          cluster_2.1.0            dbplyr_1.4.2            
##  [22] compiler_3.6.1           httr_1.4.1               backports_1.1.5         
##  [25] assertthat_0.2.1         Matrix_1.2-18            lazyeval_0.2.2          
##  [28] cli_2.0.0                acepack_1.4.1            htmltools_0.4.0         
##  [31] prettyunits_1.0.2        tools_3.6.1              gtable_0.3.0            
##  [34] glue_1.3.1               GenomeInfoDbData_1.2.2   rappdirs_0.3.1          
##  [37] BiocWorkflowTools_1.12.0 Rcpp_1.0.3               vctrs_0.2.0             
##  [40] Biostrings_2.54.0        rtracklayer_1.46.0       xfun_0.11               
##  [43] stringr_1.4.0            lifecycle_0.1.0          ensembldb_2.10.2        
##  [46] XML_3.98-1.20            zlibbioc_1.32.0          scales_1.1.0            
##  [49] hms_0.5.2                ProtGenerics_1.18.0      AnnotationFilter_1.10.0 
##  [52] RColorBrewer_1.1-2       yaml_2.2.0               curl_4.3                
##  [55] memoise_1.1.0            gridExtra_2.3            biomaRt_2.42.0          
##  [58] rpart_4.1-15             hunspell_3.0             latticeExtra_0.6-28     
##  [61] stringi_1.4.3            RSQLite_2.1.4            genefilter_1.68.0       
##  [64] checkmate_1.9.4          rlang_0.4.2              pkgconfig_2.0.3         
##  [67] commonmark_1.7           bitops_1.0-6             evaluate_0.14           
##  [70] lattice_0.20-38          purrr_0.3.3              labeling_0.3            
##  [73] GenomicAlignments_1.22.1 htmlwidgets_1.5.1        bit_1.1-14              
##  [76] tidyselect_0.2.5         magrittr_1.5             bookdown_0.16           
##  [79] R6_2.4.1                 spelling_2.1             Hmisc_4.3-0             
##  [82] DBI_1.0.0                withr_2.1.2              pillar_1.4.2            
##  [85] foreign_0.8-72           survival_3.1-8           RCurl_1.95-4.12         
##  [88] nnet_7.3-12              tibble_2.1.3             crayon_1.3.4            
##  [91] utf8_1.1.4               BiocFileCache_1.10.2     rmarkdown_1.18          
##  [94] progress_1.2.2           usethis_1.5.1            locfit_1.5-9.1          
##  [97] grid_3.6.1               data.table_1.12.8        blob_1.2.0              
## [100] git2r_0.26.1             digest_0.6.23            xtable_1.8-4            
## [103] openssl_1.4.1            munsell_0.5.0            viridisLite_0.3.0       
## [106] askpass_1.1
\end{verbatim}

\hypertarget{author-contributions}{%
\subsection{Author Contributions}\label{author-contributions}}

SL, DC, MIL and ML wrote the workflow.

\hypertarget{competing-interests}{%
\subsection{Competing interests}\label{competing-interests}}

The authors declare that they have no competing interests.

\hypertarget{funding}{%
\subsection{Funding}\label{funding}}

SL is supported by an Australian Government Research Training Program (RTP)
scholarship with a top up scholarship from CSL Limited.

\textbf{I confirm that the funders had no role in study design, data collection and
analysis, decision to publish, or preparation of the manuscript.}

\hypertarget{acknowledgements}{%
\subsection{Acknowledgements}\label{acknowledgements}}

The authors would like to thank all participants of the Bioconductor 2019 and
BiocAsia 2019 conferences who attended and provided feedback on early versions
of this workflow paper.

\renewcommand\refname{References}
  \bibliography{/Library/Frameworks/R.framework/Versions/3.6/Resources/library/fluentGenomics/vignettes/library.bib}

\end{document}
